\documentclass[paper=a4, fontsize=11pt, twoside]{scrartcl}
\usepackage{geometry} \geometry{a4paper,  bottom=33mm}

\usepackage[utf8]{inputenc}
\usepackage[german]{babel}

\usepackage[osf,sc]{mathpazo}
\usepackage{courier}
\renewcommand{\sfdefault}{uop} % ---> URW Classico Optima Clone
\renewcommand{\rmdefault}{pplj} % ---> Mathpazo Palatino
\linespread{1.05}

\usepackage{amsmath,amsfonts,amsthm} % Math packages
\usepackage{booktabs} 
\usepackage{url} 

\usepackage{siunitx}
\sisetup{locale = DE}

\usepackage{titlesec} 
\usepackage{graphicx}
\usepackage{hyperref} 
\usepackage{changepage}
\usepackage{caption}
\usepackage{enumitem}
\usepackage{minted}
\usepackage{fancyhdr}
\usepackage{multicol}
\usepackage{xspace}
\expandafter\def\expandafter\grqq\expandafter{\grqq\xspace}

% break \texttt{}
\newcommand*\justify{%
  \fontdimen2\font=0.4em% interword space
  \fontdimen3\font=0.2em% interword stretch
  \fontdimen4\font=0.1em% interword shrink
  \fontdimen7\font=0.1em% extra space
  \hyphenchar\font=`\-% allowing hyphenation
}

\pagestyle{fancy} %eigener Seitenstil
\fancyhf{} %alle Kopf- und Fußzeilenfelder bereinigen
\fancyhead[OL]{Programmieren in C++ (Tutorium)} %Kopfzeile links
\fancyhead[OC]{} %zentrierte Kopfzeile
\fancyhead[OR]{Simon Fromme, Florian Rinke} %Kopfzeile rechts
\renewcommand{\headrulewidth}{0.4pt} %obere Trennlinie
\fancyfoot[C]{\thepage} %Seitennummer

% DRAFT mode
% \usepackage{draftwatermark}

\setlength\parindent{0pt} 

\title{Programmieraufgaben 3}

\author{Simon Fromme}
\date{\normalsize\today}

\begin{document}
\vspace*{0.75\baselineskip}
\begin{center}
  \Large 3. Aufgabenblatt \\\vspace{0.5em} \large Programmieren in C++ (Tutorium)
\end{center}

% Aufgaben zu Vector (sequence containers representing arrays that can change in size)
\section*{Vector}
\begin{enumerate}
\item Machen Sie sich mit den Eigenschaften des \mintinline{c++}{vector} Containers\footnote{Sie müssen dazu die Bibliothek \texttt{vector} mit Hilfe von \mintinline{c++}{#include <vector>} einbinden } vertraut. Führen Sie dazu nacheinander folgende Schritte aus:
  \begin{itemize}
  \item Initialisieren eines (int-)Vectors mit den Werten: \texttt{11, 12, 2014, 314, 42}.
  \item Ausgeben aller Werte und der Länge des Vectors (for-Schleife)
  \item Löschen des letzten Elements,
  \item Hinzufügen der Zahl \texttt{2} an den Anfang des Vectors,
  \item Ausgabe des 3. Elements,
  \item Hinzufügen der Zahl \texttt{2818} an das Ende des Vectors,
  \item Ausgabe der Summe aller Elemente des Vectors
  \end{itemize}

   \item Schreiben Sie ein Programm, das den Mittelwert, den Median\footnote{Der Median einer Auflistung von Zahlenwerten ist der Wert, welcher an der mittleren Stelle steht, wenn man die Werte der Größe nach sortiert. Ist die Anzahl der Zahlenwerte gerade, gibt es also zwei mittlere Werte, so ist der Median der Mittelwert dieser beiden Werte.}, das Minimum und das Maximum einer Menge von Zahlen ausgibt. Lassen Sie dafür zunächst 10 float-Werte durch den Benutzer eingeben, die Sie in einem \mintinline{c++}{vector} Container speichern. Eine Erweiterung Ihres Programms auf eine beliebige Anzahl von Zahlen $n$ soll durch die Änderung einer einzelnen Variablen möglich sein. \\
Hinweis: Auf der Seite 31f in den Vorlesungsfolien (\texttt{Programmieren 3-6.pdf}) haben Sie Möglichkeiten kennengelernt, eine Liste von Zahlen zu sortieren.
\end{enumerate}

% Aufgaben zu String
\section*{String}
\begin{enumerate}[resume]
   \item
\end{enumerate}

\section*{Enum}
\begin{enumerate}[resume]
   \item
\end{enumerate}


\end{document}
