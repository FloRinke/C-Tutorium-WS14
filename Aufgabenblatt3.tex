\documentclass[paper=a4, fontsize=11pt, twoside]{scrartcl}
\usepackage{geometry} \geometry{a4paper,  bottom=33mm}

\usepackage[utf8]{inputenc}
\usepackage[german]{babel}

\usepackage[osf,sc]{mathpazo}
\usepackage{courier}
\renewcommand{\sfdefault}{uop} % ---> URW Classico Optima Clone
\renewcommand{\rmdefault}{pplj} % ---> Mathpazo Palatino
\linespread{1.05}

\usepackage{amsmath,amsfonts,amsthm} % Math packages
\usepackage{booktabs} 
\usepackage{url} 

\usepackage{siunitx}
\sisetup{locale = DE}

\usepackage{titlesec} 
\usepackage{graphicx}
\usepackage{hyperref} 
\usepackage{changepage}
\usepackage{caption}
\usepackage{enumitem}
\usepackage{minted}
\usepackage{fancyhdr}
\usepackage{multicol}
\usepackage{xspace}
\expandafter\def\expandafter\grqq\expandafter{\grqq\xspace}

% break \texttt{}
\newcommand*\justify{%
  \fontdimen2\font=0.4em% interword space
  \fontdimen3\font=0.2em% interword stretch
  \fontdimen4\font=0.1em% interword shrink
  \fontdimen7\font=0.1em% extra space
  \hyphenchar\font=`\-% allowing hyphenation
}

\pagestyle{fancy} %eigener Seitenstil
\fancyhf{} %alle Kopf- und Fußzeilenfelder bereinigen
\fancyhead[OL]{Programmieren in C++ (Tutorium)} %Kopfzeile links
\fancyhead[OC]{} %zentrierte Kopfzeile
\fancyhead[OR]{Simon Fromme, Florian Rinke} %Kopfzeile rechts
\renewcommand{\headrulewidth}{0.4pt} %obere Trennlinie
\fancyfoot[C]{\thepage} %Seitennummer

% DRAFT mode
% \usepackage{draftwatermark}

\setlength\parindent{0pt} 

\title{Programmieraufgaben 3}

\author{Simon Fromme}
\date{\normalsize\today}

\begin{document}
\vspace*{0.75\baselineskip}
\begin{center}
  \Large 3. Aufgabenblatt \\\vspace{0.5em} \large Programmieren in C++ (Tutorium)
\end{center}

% Aufgaben zu Vector (sequence containers representing arrays that can change in size)
\section*{Vector}
\begin{enumerate}
\item Machen Sie sich mit den Eigenschaften des \mintinline{c++}{vector} Containers\footnote{Sie müssen dazu die Bibliothek \texttt{vector} mit Hilfe von \mintinline{c++}{#include <vector>} einbinden } vertraut. Führen Sie dazu nacheinander folgende Schritte aus:
  \begin{itemize}
  \item Initialisieren eines (int-)Vectors mit den Werten: \texttt{11, 12, 2014, 314, 42}.
  \item Ausgeben aller Werte und der Länge des Vectors (for-Schleife)
  \item Löschen des letzten Elements,
  \item Ausgabe des 3. Elements,
  \item Hinzufügen der Zahl \texttt{2818} an das Ende des Vectors,
  \item Ausgabe der Summe aller Elemente des Vectors
  \end{itemize}

\item Schreiben Sie ein Programm, das den Mittelwert, den Median\footnote{Der Median einer Auflistung von Zahlenwerten ist der Wert, welcher an der mittleren Stelle steht, wenn man die Werte der Größe nach sortiert. Ist die Anzahl der Zahlenwerte gerade, gibt es also zwei mittlere Werte, so ist der Median der Mittelwert dieser beiden Werte.}, das Minimum und das Maximum einer Menge von Zahlen ausgibt. Lassen Sie dafür zunächst 10 float-Werte durch den Benutzer eingeben, die Sie in einem \mintinline{c++}{vector} Container speichern. Eine Erweiterung Ihres Programms auf eine beliebige Anzahl von Zahlen $n$ soll durch die Änderung einer einzelnen Variablen möglich sein. \\
Hinweis: Auf der Seite 31f in den Vorlesungsfolien (\texttt{Programmieren 3-6.pdf}) haben Sie Möglichkeiten kennengelernt, eine Liste von Zahlen zu sortieren.
\end{enumerate}

% Aufgaben zu String
\section*{String}
\begin{enumerate}[resume]
  \item Implementierung einer einfachen Verschlüsselung: Um einen Text schnell unleserlich zu machen, kann das ROT13-Verfahren\footnote{\url{http://de.wikipedia.org/wiki/ROT13}} verwendet werden, bei dem jeder Buchstabe eines Textes um 13 Stellen im Alphabet (Zahlen, Umlaute und Sonderzeichen werden außen vor gelassen) verschoben wird. \par
Schreiben Sie ein Programm, das für einen eingegebenen Text diese Umcodierung vornimmt. Testen Sieanschließend die Funktionsweise Ihres Programms, indem Sie folgenden Text entschlüsseln:  \glqq \texttt{\justify Haraqyvpure Hajnuefpurvayvpuxrvgfnagevro}\grqq. 
  \item Umwandlung in Groß-/Kleinbuchstaben: Lesen Sie einen Text ein und wandeln Sie ihn je einmal in Groß- und Kleinbuchstaben um. Geben Sie beide Texte aus. \par Lösen Sie die Aufgabe einmal von Hand mit char-Arithmethik und unter Verwendung von Funktionen der Standardbibliothek\footnote{hier ist z.B. die Bibliothek \glqq \texttt{locale}\grqq\, (\url{http://www.cplusplus.com/reference/locale/}) einen Blick Wert.}.
  \item Palindrome sind Wörter oder Sätze, die von vorne und hinten gelesen dasselbe ergeben (z.B. Anna, Lagerregal, Ein Neger mit Gazelle zagt im Regen nie). 
\par Schreiben Sie ein Programm, das für einen eingegebenen Text prüft, ob es sich dabei um ein Palindrom handelt. Satz- und Leerzeichen sollen dabei ignoriert werden.
\end{enumerate}

\section*{Struct}
\begin{enumerate}[resume]
   \item Telefonbuch: Erstellen Sie eine \mintinline{c++}{struct}-Datenstruktur, die für ein firmeninternes Telefonbuch die Daten \glqq Name\grqq, \glqq Vorname\grqq, \glqq Telefonnummer\grqq, \glqq Abteilung\grqq und \glqq Raum\grqq sinnvoll speichern kann. Überlegen Sie sich jeweils sinnvolle Datentypen.
   \item Speichern Sie folgende Datensätze als Struct (siehe vorherige Aufgabe) in einem Vektor und geben Sie sie sinnvoll formatiert aus.
\begin{table}[h!]
   \centering
   \begin{tabular}{llclr}
   Name & Vorname & Telefon & Abteilung & Raum \\    \hline 
   Vimes & Samuel & 2110 & Wache & W05 \\
   Ridcully & Mustrum  & 3305 & Forschung & T45 \\
   Vetinari & Havelock & 1001 & Geschäftsführung & P100 \\
   von Lipwig & Moist  & 1919 & Poststelle & W07 \\
   Nobbs & Cecil & 2114 & Wache & W08
   \end{tabular}
\end{table}
\end{enumerate}

\section*{Enum}
\begin{enumerate}[resume]
   \item Erweiterung des Telefonbuchs (siehe vorherige Aufgabenstellung): Da die Abteilungen sich in unserem Beispiel bekannt sind und sich nicht ändern, können sie vereinfacht gespeichert werden, um Platz zu sparen und einfachere Auswertungen zu ermöglichen. \par Erstellen Sie hierzu eine \mintinline{c++}{enum}-Datenstruktur, die die verwendeten Abteilungen als mögliche Werte beinhaltet. Ändern Sie nun den \mintinline{c++}{struct}-Datentyp des Telefonbuchs, so dass dieser den \mintinline{c++}{enum}-Datentyp verwendet, um Abteilungen zu speichern.
   \item Erweitern Sie Ihre Ausgabe so, dass bei Personen aus der Geschäftsführung \glqq \texttt{Wichtig!!}\grqq\, und bei Personen aus der Wache \glqq \texttt{Gefährlich!!}\grqq am Ende der jeweiligen Zeile ausgegeben wird (Verwenden Sie hierzu eine \mintinline{c++}{switch}/\mintinline{c++}{case}-Datenstruktur).
\end{enumerate}


\end{document}
