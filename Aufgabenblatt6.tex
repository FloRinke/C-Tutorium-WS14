\documentclass[paper=a4, fontsize=11pt, twoside]{scrartcl}
\usepackage{geometry} \geometry{a4paper,  bottom=33mm}

\usepackage[utf8]{inputenc}
\usepackage[german]{babel}

\usepackage[osf,sc]{mathpazo}
\usepackage{courier}
\renewcommand{\sfdefault}{uop} % ---> URW Classico Optima Clone
\renewcommand{\rmdefault}{pplj} % ---> Mathpazo Palatino
\linespread{1.05}

\usepackage{amsmath,amsfonts,amsthm} % Math packages
\usepackage{booktabs}
\usepackage{url}

\usepackage{siunitx}
\sisetup{locale = DE}

\usepackage{titlesec}
\usepackage{graphicx}
\usepackage{hyperref}
\usepackage{changepage}
\usepackage{caption}
\usepackage{enumitem}
\usepackage{minted}
\usepackage{fancyhdr}
\usepackage{multicol}
\usepackage{xspace}
\expandafter\def\expandafter\grqq\expandafter{\grqq\xspace}

% break \texttt{}
\newcommand*\justify{%
  \fontdimen2\font=0.4em% interword space
  \fontdimen3\font=0.2em% interword stretch
  \fontdimen4\font=0.1em% interword shrink
  \fontdimen7\font=0.1em% extra space
  \hyphenchar\font=`\-% allowing hyphenation
}

\pagestyle{fancy} %eigener Seitenstil
\fancyhf{} %alle Kopf- und Fußzeilenfelder bereinigen
\fancyhead[OL]{Programmieren in C++ (Tutorium)} %Kopfzeile links
\fancyhead[OC]{} %zentrierte Kopfzeile
\fancyhead[OR]{Simon Fromme, Florian Rinke, Lars Franke} %Kopfzeile rechts
\renewcommand{\headrulewidth}{0.4pt} %obere Trennlinie
\fancyfoot[C]{\thepage} %Seitennummer

% DRAFT mode
% \usepackage{draftwatermark}

\setlength\parindent{0pt}

\title{Programmieraufgaben 6}

\author{Simon Fromme}
\date{\normalsize\today}

\begin{document}
\vspace*{0.75\baselineskip}
\begin{center}
  \Large 6. Aufgabenblatt \\\vspace{0.5em} \large Programmieren in C++ (Tutorium)
\end{center}

\section*{Pointer}
\begin{enumerate}
  \item ToDo: Adresse und Inhalt von Elementen
  \item ToDo: const Pointer
  \item RFC: Pointer-Arithmethik?
\end{enumerate}

\section*{Arrays}
\begin{enumerate}
  \item ToDo: Initialisierung und Zugriff
  \item ToDo: Adressierung per [] und *
  \item ToDo: Arrays als Parameter für Funktionen
\end{enumerate}


\section*{C-Strings}
\begin{enumerate}
  \item ToDo: Initialisierung und Elementweiser Zugriff
  \item ToDo: Kopieren
  \item RFC: Sicherheitsprobleme durch Längenverletzung?
\end{enumerate}

\end{document}

% neue Konzepte seit dem letzten Aufgabenblatt:
% - Pointer
%   * Adresse und Inhalt
%   * const pointer und pointer auf const
% - Arrays
%   * Adressierung als Array und per Pointer
%   * Intialisierung (Liste, Anzahl)
% - C-Strings
%   * Intialisierung
%   * Elementweiser Zugriff
%   * Kopieren

