\documentclass[paper=a4, fontsize=11pt, twoside]{scrartcl}
\usepackage{geometry} \geometry{a4paper,  bottom=33mm}

\usepackage[utf8]{inputenc}
\usepackage[german]{babel}

\usepackage[osf,sc]{mathpazo}
\usepackage{courier}
\renewcommand{\sfdefault}{uop} % ---> URW Classico Optima Clone
\renewcommand{\rmdefault}{pplj} % ---> Mathpazo Palatino
\linespread{1.05}

\usepackage{amsmath,amsfonts,amsthm} % Math packages
\usepackage{booktabs}
\usepackage{url}

\usepackage{siunitx}
\sisetup{locale = DE}

\usepackage{titlesec}
\usepackage{graphicx}
\usepackage{hyperref}
\usepackage{changepage}
\usepackage{caption}
\usepackage{enumitem}
\usepackage{minted}
\usepackage{fancyhdr}
\usepackage{multicol}
\usepackage{xspace}
\expandafter\def\expandafter\grqq\expandafter{\grqq\xspace}

% break \texttt{}
\newcommand*\justify{%
  \fontdimen2\font=0.4em% interword space
  \fontdimen3\font=0.2em% interword stretch
  \fontdimen4\font=0.1em% interword shrink
  \fontdimen7\font=0.1em% extra space
  \hyphenchar\font=`\-% allowing hyphenation
}

\pagestyle{fancy} %eigener Seitenstil
\fancyhf{} %alle Kopf- und Fußzeilenfelder bereinigen
\fancyhead[OL]{Programmieren in C++ (Tutorium)} %Kopfzeile links
\fancyhead[OC]{} %zentrierte Kopfzeile
\fancyhead[OR]{Simon Fromme, Florian Rinke, Lars Franke} %Kopfzeile rechts
\renewcommand{\headrulewidth}{0.4pt} %obere Trennlinie
\fancyfoot[C]{\thepage} %Seitennummer

% DRAFT mode
% \usepackage{draftwatermark}

\setlength\parindent{0pt}

\title{Programmieraufgaben 6}

\author{Simon Fromme}
\date{\normalsize\today}

\begin{document}
\vspace*{0.75\baselineskip}
\begin{center}
  \Large 6. Aufgabenblatt \\\vspace{0.5em} \large Programmieren in C++ (Tutorium)
\end{center}

\section*{Pointer}
\begin{enumerate}
  %ToDo: Adresse und Inhalt von Elementen
  \item 
  %ToDo: const Pointer
  \item 
  %RFC: Pointer-Arithmethik?
  \item 
\end{enumerate}

\section*{Arrays}
\begin{enumerate}
  %Initialisierung und Zugriff
  \item Legen Sie ein Array \mintinline{c++}{int data[5];} an und lassen Sie sich den Inhalt der Elemente ausgeben. Testen und Vergleichen Sie die Ausgabe im Debug- und Release-Modus. Vergleichen Sie Ihre Ergebnisse mit denen Ihrer Kommilitonen. Was stellen Sie fest?
  \item Zur Erfassung von 1000 Messwerten (Fließkommazahlen) wird ein Datenspeicher benötigt. Legen Sie ein geeignetes Array an und stellen Sie sicher, dass es jedes Element den Wert 0 enthält. Wie können Sie dies am einfachsten erreichen?
  %ToDo: Adressierung per [] und *
  \item 
  %Arrays als Parameter für Funktionen
  \item Legen Sie ein \mintinline{c++}{int[]}-Feld mit einigen Werten an. Geben Sie den Inhalt und die Größe des Feldes \footnote{hierbei hilft der \mintinline{c++}{sizeof}-Operator} aus. Anschließend übergeben Sie das Array an eine Funktion \mintinline{c++}{checkArray(int[] data)} und lassen diese das gleiche machen. Welches Problem stellen Sie fest? Wie müssen die die Funktionsdeklaration erweitern um dieses Problem zu lösen?
\end{enumerate}


\section*{C-Strings}
\begin{enumerate}
  %ToDo: Initialisierung und Elementweiser Zugriff
  \item Auf Übungsblatt 3 wurde geprüft, ob ein Text ein Palindrom ist (also von vorne und hinten gelesen das gleiche ergibt). Schreiben Sie erneut ein Programm, das für ein eingegebenes Wort überprüft, ob es sich um ein Palindrom handelt. Verwenden Sie diesmal statt strings aber \mintinline{c++}{char[]} zum Speichern der Daten. Um den Testablauf zu beschleunigen, kann ein geeigneter Text (z.B. \glqq Lagerregal\grqq) im Programmcode abgelegt werden anstatt ihn jedesmal von der Tastatus einzulesen. (Zusatzaufgabe: Erweitern Sie Ihr Programm, so dass auch ganze Sätze überprüft werden können. Leerzeichen sollen dabei ignoriert werden, \glqq Ein Neger mit Gazelle zagt im Regen nie\grqq\, ist beispielsweise ein gültiges Palindrom.)
  %ToDo: Kopieren
  \item 
  %RFC: Sicherheitsprobleme durch Längenverletzung?
  \item 
\end{enumerate}

\end{document}

% neue Konzepte seit dem letzten Aufgabenblatt:
% - Pointer
%   * Adresse und Inhalt
%   * const pointer und pointer auf const
% - Arrays
%   * Adressierung als Array und per Pointer
%   * Intialisierung (Liste, Anzahl)
% - C-Strings
%   * Intialisierung (done)
%   * Elementweiser Zugriff (done)
%   * Kopieren

