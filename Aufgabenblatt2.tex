\documentclass[paper=a4, fontsize=11pt, twoside]{scrartcl}
\usepackage{geometry} \geometry{a4paper,  bottom=33mm}

\usepackage[utf8]{inputenc}
\usepackage[german]{babel}

\usepackage[osf,sc]{mathpazo}
\usepackage{courier}
\renewcommand{\sfdefault}{uop} % ---> URW Classico Optima Clone
\renewcommand{\rmdefault}{pplj} % ---> Mathpazo Palatino
\linespread{1.05}

\usepackage{amsmath,amsfonts,amsthm} % Math packages
\usepackage{booktabs} 
\usepackage{url} 

\usepackage{siunitx}
\sisetup{locale = DE}

\usepackage{titlesec} 
\usepackage{graphicx}
\usepackage{hyperref} 
\usepackage{changepage}
\usepackage{caption}
\usepackage{enumitem}
\usepackage{minted}
\usepackage{fancyhdr}

\pagestyle{fancy} %eigener Seitenstil
\fancyhf{} %alle Kopf- und Fußzeilenfelder bereinigen
\fancyhead[OL]{Programmieren in C++ (Tutorium)} %Kopfzeile links
\fancyhead[OC]{} %zentrierte Kopfzeile
\fancyhead[OR]{Simon Fromme, Florian Rinke} %Kopfzeile rechts
\renewcommand{\headrulewidth}{0.4pt} %obere Trennlinie
\fancyfoot[C]{\thepage} %Seitennummer

% DRAFT mode
% \usepackage{draftwatermark}

\setlength\parindent{0pt} 

\title{Programmieraufgaben 2}

\author{Simon Fromme}
\date{\normalsize\today}

\begin{document}
\vspace*{0.75\baselineskip}
\begin{center}
  \Large 2. Aufgabenblatt \\\vspace{0.5em} \large Programmieren in C++ (Tutorium)
\end{center}

% Aufgaben zu if/then/else,  switch, expr ? then : else
\section*{Kontrollstrukturen}
\begin{enumerate}
\item Schreibe Sie ein Programm, das nach Eingabe von zwei Zahlen $a$ und $b$, deren Maximum $\max(a,b)$ ausgibt. Nutzen Sie dabei zunächst eine reguläre \texttt{if}/\texttt{then}/\texttt{else}-Anweisung und modifizieren Sie Ihr Programm anschließend so, dass Sie stattdessen den Bedingungsoperator\footnote{Sie haben den Bedingungsoperator \mintinline{c++}{?} auf Seite 13 der Vorlesungsfolien (\texttt{Programmieren 3-5.pdf}) kennengelernt.} \mintinline{c++}{?} verwenden. 
\item Schreiben Sie ein Programm, das nach Eingabe einer Zahl 1-7 den dazugehörigen Wochentag (Montag-Sonntag) ausgibt. Falls keine gültige Zahl eingegeben wurde, soll das Programm den Text ``\texttt{Kein gültiger Wochentag!}'' ausgeben.
\end{enumerate}

% Aufgaben zu Schleifen (while, do-while, for, break-Anweisung, continue-Anweisung
\section*{Schleifen}
\begin{enumerate}[resume]
 \item Im Jahr 1582 wurde von  Papst Gregor XIII. der Gregorianische Kalender und damit die bis heute geltende Schaltjahresreglung eingeführt. Diese lautet wie folgt:
   \begin{itemize}
   \item Jahre, deren Jahreszahlen durch 4 teilbar sind, sind Schaltjahre (z.B. 1992), 
   \item Jahre, deren Jahreszahlen durch 100 teilbar sind, sind keine Schaltjahre (z.B. 1900), 
   \item Jahre, deren Jahreszahlen durch 400 teilbar sind, sind wiederum Schaltjahre (z.B. 2000).
   \end{itemize}
   Schreiben Sie ein Programm, das die Anzahl der Schaltjahre seit Einführung des Gregorianischen Kalenders berechnet.
\end{enumerate}

\end{document}
