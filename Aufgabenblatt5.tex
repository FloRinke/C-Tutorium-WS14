\documentclass[paper=a4, fontsize=11pt, twoside]{scrartcl}
\usepackage{geometry} \geometry{a4paper,  bottom=33mm}

\usepackage[utf8]{inputenc}
\usepackage[german]{babel}

\usepackage[osf,sc]{mathpazo}
\usepackage{courier}
\renewcommand{\sfdefault}{uop} % ---> URW Classico Optima Clone
\renewcommand{\rmdefault}{pplj} % ---> Mathpazo Palatino
\linespread{1.05}

\usepackage{amsmath,amsfonts,amsthm} % Math packages
\usepackage{booktabs} 
\usepackage{url} 

\usepackage{siunitx}
\sisetup{locale = DE}

\usepackage{titlesec} 
\usepackage{graphicx}
\usepackage{hyperref} 
\usepackage{changepage}
\usepackage{caption}
\usepackage{enumitem}
\usepackage{minted}
\usepackage{fancyhdr}
\usepackage{multicol}
\usepackage{xspace}
\expandafter\def\expandafter\grqq\expandafter{\grqq\xspace}

% break \texttt{}
\newcommand*\justify{%
  \fontdimen2\font=0.4em% interword space
  \fontdimen3\font=0.2em% interword stretch
  \fontdimen4\font=0.1em% interword shrink
  \fontdimen7\font=0.1em% extra space
  \hyphenchar\font=`\-% allowing hyphenation
}

\pagestyle{fancy} %eigener Seitenstil
\fancyhf{} %alle Kopf- und Fußzeilenfelder bereinigen
\fancyhead[OL]{Programmieren in C++ (Tutorium)} %Kopfzeile links
\fancyhead[OC]{} %zentrierte Kopfzeile
\fancyhead[OR]{Simon Fromme, Florian Rinke, Lars Franke} %Kopfzeile rechts
\renewcommand{\headrulewidth}{0.4pt} %obere Trennlinie
\fancyfoot[C]{\thepage} %Seitennummer

% DRAFT mode
% \usepackage{draftwatermark}

\setlength\parindent{0pt} 

\title{Programmieraufgaben 3}

\author{Simon Fromme}
\date{\normalsize\today}

\begin{document}
\vspace*{0.75\baselineskip}
\begin{center}
  \Large 5. Aufgabenblatt \\\vspace{0.5em} \large Programmieren in C++ (Tutorium)
\end{center}

% Aufgaben zu Vector (sequence containers representing arrays that can change in size)
%     - Modularisierung
%     - Überladen
%     - Prototypen
%     - inline Functions
%     - Übergabe per Referenz

\section*{Eingabe \& Ausgabe}
\begin{enumerate}
\item foo
% TODO: Aufgabe zur Ein- und Ausgabe erstellen
\end{enumerate}

\section*{Funktionen}\label{sec:funktionen}
In dieser Aufgabe müssen einige Funktionen geschrieben werden. Verwenden Sie für die folgenden Abschnitte eine einzelne Programmdatei und testen sie die jeweiligen Funktionen, indem Sie diese von der \mintinline{c++}{main}-Methode aus aufrufen.
\begin{enumerate}[resume]
\item Schreiben Sie ein Programm, das die Fakultät $n!$ einer vorgegbenen Zahl $n$ berechnet. Definieren Sie dazu eine Funktion \mintinline{c++}{int fak(n)}. 
  \item Schreiben Sie ein Programm, welches die Summe aller Quadratzahlen $\sum\limits_{i=1}^{n}i^2$ bis zu einer vorgegbenen Zahl $n$ berechnet. Definieren Sie dazu eine Funktion \\ \mintinline{c++}{int sumquad(n)}.
  \item Schreiben Sie eine Funktion \mintinline{c++}{division(a,b)}, die den Quotienten $a/b$ zweier \mintinline{c++}{float}-Zahlen berechnet. Wenn die Division nicht möglich ist, soll die Zahl \texttt{0} zurück gegeben werden.\label{item:division}
\end{enumerate}

\section*{Überladen von Funktionen}\label{sec:uberl-von-funktionen}
\begin{enumerate}[resume]
\item Die Funktion \mintinline{c++}{division(a,b)} soll nun bei ganzzahligen Parametern (Datentyp: \mintinline{c++}{int}) das Ergebnis der ganzzahligen Division zurückgeben\footnote{Für zwei ganze Zahlen $a$ und $b$ ist das Ergebnis der ganzzahligen Division $(a/b)_{\mathrm{int}}$ definiert als $(a/b)_{\mathrm{int}}:= \max\{z \in \mathbb{R}:z\le a/b \}$, wobei $a/b$ die ``normale'' Division ist.}. Überladen Sie dafür die in Aufgabe \ref{item:division} geschriebene Funktion so, dass für ganzzahlige Parameter der gewünschte Wert ausgegeben wird. 
\end{enumerate}

\section*{Modularisierung}\label{sec:modularisierung}
\begin{enumerate}[resume]
\item Um eine bessere Übersichtlichkeit des Programmcodes zu gewährleisten, trennt man häufig die Deklaration (s.g. Prototyp) von der Implementation einer Funktion. \par
Modifizieren Sie ihre bisher erstellten Funktionen so, dass die jeweilige Funktion vor der \mintinline{c++}{main}-Methode deklariert, allerdings erst am Ende der Datei implementiert werden.
\end{enumerate}

\section*{Übergabe nach Name/Referenz}
\begin{enumerate}[resume]
\item Machen Sie sich den Unterschied zwischen dem Übergeben eines Parameters durch seinen Namen bzw. durch seine Referenz (vorangestelltes \textrm{&}-Symbol klar. \par
  Schreiben Sie dazu zunächste eine Funktion \mintinline{c++}{int inc(n)}, die den Wert $n+1$ zurückgibt, ohne dabei den Parameter im aufrufenden Kontext zu verändern. \\
  Wie muss die Fuktion mpdifiziert werden, damit der Parameter im aufrufenden Kontext durch die Funktion verändert wird? 
\end{enumerate}

\end{document}


% TODO: Aufgabe zu Ein- und Ausgabe erarbeiten (ähnlich vom Weihnachtsübungsblatt)

% neue Konzepte seit dem letzten Aufgabenblatt:
% - Modularisierung (done)
% - Überladen (done)
% - Prototypen (done)
% - inline Functions (TODO: Nachlesen in der Referenz)
% - Übergabe per Referenz (done)